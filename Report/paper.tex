
%%%%%%%%%%%%%%%%%%%%%%% file typeinst.tex %%%%%%%%%%%%%%%%%%%%%%%%%
%
% This is the LaTeX source for the instructions to authors using
% the LaTeX document class 'llncs.cls' for contributions to
% the Lecture Notes in Computer Sciences series.
% http://www.springer.com/lncs       Springer Heidelberg 2006/05/04
%
% It may be used as a template for your own input - copy it
% to a new file with a new name and use it as the basis
% for your article.
%
% NB: the document class 'llncs' has its own and detailed documentation, see
% ftp://ftp.springer.de/data/pubftp/pub/tex/latex/llncs/latex2e/llncsdoc.pdf
%
%%%%%%%%%%%%%%%%%%%%%%%%%%%%%%%%%%%%%%%%%%%%%%%%%%%%%%%%%%%%%%%%%%%


%%\documentclass[runningheads,a4paper]{llncs}
\documentclass[runningheads]{llncs}

\usepackage{amssymb}
\setcounter{tocdepth}{3}
\usepackage{graphicx}

\usepackage{amsmath}
\usepackage{listings}

\usepackage{verbatim}
\usepackage{longtable}
\usepackage[table]{xcolor}
\usepackage{tabularx, booktabs}
\usepackage{multirow}
\usepackage{epsfig}
\usepackage{pdflscape} 

%\usepackage{algorithm2e}
%\usepackage[linesnumbered]{algorithm2e}
%\usepackage[linesnumbered,ruled]{algorithm2e}
\usepackage[linesnumbered,ruled,vlined]{algorithm2e}

\newcolumntype{Y}{>{\centering\arraybackslash}X}

% \renewcommand{\ttdefault}{pcr} % makes bold available in a fixed-width font (use )

\lstMakeShortInline[basicstyle=\small]|

\lstset{
%    numbers=left
  , numberstyle=\tiny, stepnumber=5, firstnumber=1, numbersep=5pt
  , language=[Sharp]C
  , breaklines=true
  , tabsize=4
  , columns=fullflexible
  , keepspaces=true
  , basicstyle=\fontfamily{lmtt}\selectfont  % fixed-width font (better, but too wide for 2-column layout)
  , captionpos=t
  , frame=single
}

%%\usepackage{url}
%%\urldef{\mailsa}\path|{alfred.hofmann, ursula.barth, ingrid.haas, frank.holzwarth,|
%%\urldef{\mailsb}\path|anna.kramer, leonie.kunz, christine.reiss, nicole.sator,|
%%\urldef{\mailsc}\path|erika.siebert-cole, peter.strasser, lncs}@springer.com|    
%%\newcommand{\keywords}[1]{\par\addvspace\baselineskip
%%\noindent\keywordname\enspace\ignorespaces#1}

\begin{document}

\mainmatter  % start of an individual contribution

% first the title is needed
\title{An Algorithm for In-Situ Bus Scheduling and Passenger State Update}

% a short form should be given in case it is too long for the running head
\titlerunning{In-Situ Bus Scheduling}

% the name(s) of the author(s) follow(s) next
%
% NB: Chinese authors should write their first names(s) in front of
% their surnames. This ensures that the names appear correctly in
% the running heads and the author index.
%
\author{Yung-Fu Chen \and Alan Weide}
\authorrunning{Yung-Fu Chen \and et al.}

% the affiliations are given next; don't give your e-mail address
% unless you accept that it will be published

\institute{The Ohio State University, Columbus OH 43221, USA\\
\email{chen.6655@osu.edu, weide.3@osu.edu}}

%%\toctitle{Lecture Notes in Computer Science}
%%\tocauthor{Authors' Instructions}
\maketitle

% Avoid using chapter/section number in numbering a listing
\renewcommand\thelstlisting{\arabic{lstlisting}}

% Sections

\section{Introduction}
Your final project report should include:
1. A write-up describing 

(a) the design rationale of the module(s) you developed, along with the algorithm you have designed/used and a detailed API

(b) instructions and illustration of use, in terms of a README file and/or user guide, 

(c) how your implementation has been tested, 

(d) performance of your implementation,

(e) assumptions or limitations of your design,

(f) any omissions or errors in your implementation, and 

(g) any bugs or weaknesses you have found in the algorithm/protocol. (Only if some of these do not apply to your project should you omit them.)

2. Program sources and executable (including any test programs). Credit will be reserved for the quality of documentation of your programs.

3. Script of a session showing the program under test.

\subsection{Problem Specification}
\subsection{Bus Stop Functions}

\section{Algorithm}
\subsection{Request Assignment Algorithm}

\section{Implementation}
\subsection{.NOW motes vs. Emulator}
\subsection{Software Architecture}
\subsection{Testing and Evaluation}
\subsection{Limitations, Challenges, and Omissions}
\begin{itemize}
  \item Request Generation
  \item Search for nearby buses: BFS vs. Broadcast
  \item No time synchronization---instead, use Emulator's absolute time, which is based on the host's time, which is the same for all nodes
\end{itemize}
\subsection{Using the Smart-Bus Application}

\section{Conclusion}

\newpage
\appendix
\section{Source Code}

% \bibliographystyle{abbrv}
% \bibliography{merged-bibliography,siv-bibliography,bibliography_3,bibliography_lastminute,Xiong2017Abstract} % ,weide-bibliography}

\end{document}
